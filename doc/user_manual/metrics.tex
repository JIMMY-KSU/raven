
\section{Metrics}
\label{sec:Metrics}

\newcommand{\metrictypeI}[3]
{
  This metric interface directly with the metric available within \textit{#1}.
  The specifications of this metric must be defined within the XML block \xmlNode{#2}.
  This XML node needs to contain the following subnode:

  \begin{itemize}
    \item \xmlNode{metricType}\texttt{#3}\xmlNode{/metricType}, \xmlDesc{vertical bar (\texttt{|}) separated
      string, required field}.
  \end{itemize}

}
\newcommand{\metrictypeII}[3]
{
  \metrictypeI{#1}{#2}{#3}

  In addition to this XML subnode, the users can also specify the weights for given metric:
  \begin{itemize}
    \item \xmlNode{w}, \xmlDesc{python list, optional parameter}, the weights for each value in \textit{u}
      and \textit{v}. Default is None, which gives each value a weight of 1.0.
  \end{itemize}
}

\newcommand{\metrictypeIII}[3]
{
  \metrictypeI{#1}{#2}{#3}

  In addition to this XML subnode, the users can also specify the weights for given metric:
  \begin{itemize}
    \item \xmlNode{sample\_weight}, \xmlDesc{python list, optional parameter}, the weights for each value in \textit{u}
      and \textit{v}. Default is None, which gives each value a weight of 1.0.
  \end{itemize}
}

The Metrics block allows the user to specify the similarity/dissimilarity metrics to be used for other
RAVEN entities, such as \textbf{PostProcessors}, and \textbf{HybridModel}.

In the RAVEN input file these metrics are defined as follows:
\begin{lstlisting}[style=XML]
<Simulation>
  ...
  <Metrics>
    ...
    <MetricID name='metricName'>
      ...
     <param1>value</param1>
      ...
    </MetricID>
    ...
  </Metrics>
  ...
</Simulation>
\end{lstlisting}

The metrics, that are available in RAVEN, can be categorized into several main classes:
\begin{itemize}
  \item \textbf{Paired Distance Metric}, distance metrics between two variables $u$ and $v$, such as \xmlString{euclidean},
    \xmlString{manhattan}, \xmlString{minkowski} and so on.
  \item \textbf{Regression Metric}, measure the regression performance, such as \xmlString{mean\_squared\_error},
    \xmlString{r2\_score}, \xmlString{explained\_variance\_score} and \xmlString{mean\_absolute\_error}.
  \item \textbf{Boolean Metric}, distance metrics between two boolean variables $u$ and $v$, such as
    \xmlString{dice}, \xmlString{hamming}, \xmlString{yule} and so on.
  \item \textbf{Other metric}, such as \xmlString{DTW}.
\end{itemize}

The valid \textbf{MetricID}s are: \xmlNode{SKL}, \xmlNode{ScipyMetric}, \xmlNode{DTW}, \xmlNode{CDFAreaDifference},
and \xmlNode{PDFCommonArea}. This XML node requires the following attributes:
\begin{itemize}
  \item \xmlAttr{name}, \xmlDesc{required string attribute}, user-defined name of this metric. \nb As with other
    objects, this name can be used to refer to this specific entity from other input blocks in the XML.
\end{itemize}

In RAVEN, lots of metrics are just interfaces directly with metrics available within \textbf{Scipy} and
\textbf{SciKit-Learn}. In this case, the algorithm for the metrics is choosen by the subnode \xmlNode{metricType}
under the parent node \xmlNode{SKL} (metric from SciKit-Learn) or \xmlNode{ScipyMetric} (metric from Scipy). For
example, \xmlNode{metricType}\xmlString{paired\_distance|euclidean}\xmlNode{/metricType}.

In the following sub-sections, the input requirements for all of the metrics are presented in the following sections.

%%%%%%%%%%%%%%%%%%%%%%%%%%%%%%%%%%%%%%%%%%%%%%%%%%%%%%%%%%%%%%%%%%%%%%%%%%%%%%%%%%%%%%%%%%%%%%%%%%%
% paired_distance metric
%%%%%%%%%%%%%%%%%%%%%%%%%%%%%%%%%%%%%%%%%%%%%%%%%%%%%%%%%%%%%%%%%%%%%%%%%%%%%%%%%%%%%%%%%%%%%%%%%%%
\subsection{Paired Distance Metric}
\label{subsection:pairedDistance}

\subsubsection{Euclidean}
This metric compute the paired euclidean distances between $u$ and $v$, i.e.
\begin{equation}
  {||u-v||}_2
\end{equation}

\metrictypeI{SciKit-Learn}{SKL}{paired\_distance|euclidean}

\subsubsection{Cosine}
This metric computes the paired cosine distances between $u$ and $v$, i.e.
\begin{equation}
  1 - \frac{u \cdot v}{||u||_2 ||v||_2}
\end{equation}
where $u \cdot v$ is the dot product of $u$ and $v$

\metrictypeI{SciKit-Learn}{SKL}{paired\_distance|cosine}

\subsubsection{Manhattan}
This metric computes the L1 distances between $u$ and $v$, i.e.
\begin{equation}
  \sum_i {\left| u_i - v_i \right|}
\end{equation}

\metrictypeI{SciKit-Learn}{SKL}{paired\_distance|manhattan}

\subsubsection{Braycurtis}
This metric computes the Bray-Curtis distances between $u$ and $v$, i.e.
\begin{equation}
  \sum{|u_i-v_i|} / \sum{|u_i+v_i|}
\end{equation}
The Bray-Curtis distance is in the range $[0, 1]$.
\metrictypeII{Scipy}{ScipyMetric}{paired\_distance|braycurtis}

\subsubsection{Canberra}
This metric computes the Canberra distance between $u$ and $v$, i.e.
\begin{equation}
  d(u,v) = \sum_i \frac{|u_i-v_i|}{|u_i|+|v_i|}
\end{equation}

\metrictypeII{Scipy}{ScipyMetric}{paired\_distance|canberra}

\subsubsection{Correlation}
This metric computes the correlation distance between $u$ and $v$, i.e.
\begin{equation}
  1 - \frac{(u - \bar{u}) \cdot (v - \bar{v})}{{||(u - \bar{u})||}_2 {||(v - \bar{v})||}_2}
\end{equation}
where $\bar{u}$ is the mean of the elements of $u$

\metrictypeII{Scipy}{ScipyMetric}{paired\_distance|correlation}

\subsubsection{Minkowski}
This metric computes the Minkowski distance between $u$ and $v$, i.e.
\begin{equation}
  {||u-v||}_p = (\sum{|u_i - v_i|^p})^{1/p}
\end{equation}

\metrictypeII{Scipy}{ScipyMetric}{paired\_distance|minkowski}

\begin{itemize}
  \item \xmlNode{p}, \xmlDesc{float, required field}, value for the parameter $p$
\end{itemize}

In the RAVEN input file, these metrics are defined as follows:
\begin{lstlisting}[style=XML]
<Simulation>
  ...
  <Metrics>
    <SKL name="euclidean">
        <metricType>paired_distance|euclidean</metricType>
    </SKL>
    <SKL name="cosine">
        <metricType>paired_distance|cosine</metricType>
    </SKL>
    <SKL name="manhattan">
        <metricType>paired_distance|manhattan</metricType>
    </SKL>
    <ScipyMetric name="braycurtis">
      <metricType>paired_distance|braycurtis</metricType>
    </ScipyMetric>
    <ScipyMetric name="canberra">
        <metricType>paired_distance|canberra</metricType>
    </ScipyMetric>
    <ScipyMetric name="correlation">
        <metricType>paired_distance|correlation</metricType>
    </ScipyMetric>
    <ScipyMetric name="minkowski">
        <metricType>paired_distance|minkowski</metricType>
        <p>5</p>
        <w>[0.4, 0.3, 0.2, 0.1, 0.5]</w>
    </ScipyMetric>
  </Metrics>
  ...
</Simulation>
\end{lstlisting}

%%%%%%%%%%%%%%%%%%%%%%%%%%%%%%%%%%%%%%%%%%%%%%%%%%%%%%%%%%%%%%%%%%%%%%%%%%%%%%%%%%%%%%%%%%%%%%%%%%%
% regression  metric
%%%%%%%%%%%%%%%%%%%%%%%%%%%%%%%%%%%%%%%%%%%%%%%%%%%%%%%%%%%%%%%%%%%%%%%%%%%%%%%%%%%%%%%%%%%%%%%%%%%
\subsection{Regression Metric}
\label{subsection:regression}

\subsubsection{Explained variance score}
This metric computes the explained variance regression score, i.e.
\begin{equation}
  1.0 - \frac{Var[u-v]}{Var[u]}
\end{equation}
The best possible score is 1.0, lower values are worse.

\metrictypeIII{Scikit-Learn}{SKL}{regression|explained\_variance\_score}

\subsubsection{Mean absolute error}
This metric computes mean absolute error, a risk metric corresponding to the expected value of the absolute
error loss or \textit{l1}-norm loss.
\begin{equation}
  \frac{1}{n_{samples}}\sum_{i=0}^{n_{samples}-1}|u_i-v_i|
\end{equation}

\metrictypeIII{Scikit-Learn}{SKL}{regression|mean\_absolute\_error}

\subsubsection{Mean squared error}
This metric computes mean square error, a risk metric corresponding to the expected value of the
squared error or loss.
\begin{equation}
  \frac{1}{n_{samples}}\sum_{i=0}^{n_{samples}-1}(u_i-v_i)^2
\end{equation}

\metrictypeIII{Scikit-Learn}{SKL}{regression|mean\_squared\_error}

\subsubsection{R2 score}
This metric computes the coefficient of determination, i.e.
\begin{equation}
  1.0-\frac{\sum_{i=0}^{n_{samples}-1}(u_i-v_i)^2}{\sum_{i=0}^{n_{samples}-1}(u_i-mean[u])^2}
\end{equation}
It provides a measure of how well future samples are likely to be predicted by the model.
Best possible score is 1.0 and it can be negative.

\metrictypeIII{Scikit-Learn}{SKL}{regression|r2\_score}

In the RAVEN input file, these metrics are defined as follows:
\begin{lstlisting}[style=XML]
<Simulation>
  ...
  <Metrics>
    <SKL name="explained_variance_score">
        <metricType>regression|explained_variance_score</metricType>
        <sample_weight>[0.1,0.1,0.1,0.05,0.05]</sample_weight>
    </SKL>
    <SKL name="mean_absolute_error">
        <metricType>regression|mean_absolute_error</metricType>
        <sample_weight>[0.1,0.1,0.1,0.05,0.05]</sample_weight>
    </SKL>
    <SKL name="r2_score">
        <metricType>regression|r2_score</metricType>
        <sample_weight>[0.1,0.1,0.1,0.05,0.05]</sample_weight>
    </SKL>
    <SKL name="mean_squared_error">
        <metricType>regression|mean_squared_error</metricType>
        <sample_weight>[0.1,0.1,0.1,0.05,0.05]</sample_weight>
    </SKL>
  </Metrics>
  ...
</Simulation>
\end{lstlisting}

%%%%%%%%%%%%%%%%%%%%%%%%%%%%%%%%%%%%%%%%%%%%%%%%%%%%%%%%%%%%%%%%%%%%%%%%%%%%%%%%%%%%%%%%%%%%%%%%%%%
% boolean  metric
%%%%%%%%%%%%%%%%%%%%%%%%%%%%%%%%%%%%%%%%%%%%%%%%%%%%%%%%%%%%%%%%%%%%%%%%%%%%%%%%%%%%%%%%%%%%%%%%%%%
\subsection{Boolean Metric}
\label{subsection:boolean}

\subsubsection{Dice}
This metric computes the Dice dissimilarity between two boolean variables $u$ and $v$
\begin{equation}
  \frac{c_{TF} + c_{FT}}{2c_{TT} + c_{FT} + c_{TF}}
\end{equation}
where $c_{ij}$ is the number of occurrences of $\mathtt{u[k]} = i$ and $\mathtt{v[k]} = j$ for $k < n$

\metrictypeII{Scipy}{ScipyMetric}{boolean|dice}

\subsubsection{Hamming}
This metric computes the Hamming distance between two boolean variables $u$ and $v$, i.e.
\begin{equation}
  \frac{c_{01} + c_{10}}{n}
\end{equation}
where $c_{ij}$ is the number of occurrences of $\mathtt{u[k]} = i$ and $\mathtt{v[k]} = j$ for $k < n$

\metrictypeII{Scipy}{ScipyMetric}{boolean|hamming}

\subsubsection{Jaccard}
This metric computes the Jaccard-Needham dissimilarity distance between two boolean variables $u$ and $v$, i.e.
\begin{equation}
  \frac{c_{TF} + c_{FT}}{c_{TT} + c_{FT} + c_{TF}}
\end{equation}
where $c_{ij}$ is the number of occurrences of $\mathtt{u[k]} = i$ and $\mathtt{v[k]} = j$ for $k < n$

\metrictypeII{Scipy}{ScipyMetric}{boolean|jaccard}

\subsubsection{Kulsinski}
This metric computes the Kulsinski dissimilarity distance between two boolean variables $u$ and $v$, i.e.
\begin{equation}
  \frac{c_{TF} + c_{FT} - c_{TT} + n}{c_{FT} + c_{TF} + n}
\end{equation}
where $c_{ij}$ is the number of occurrences of $\mathtt{u[k]} = i$ and $\mathtt{v[k]} = j$ for $k < n$

\metrictypeII{Scipy}{ScipyMetric}{boolean|kulsinski}

\subsubsection{Rogerstanimoto}
This metric computes the Rogers-Tanimoto dissimilarity distance between two boolean variables $u$ and $v$, i.e.
\begin{equation}
  \frac{R}{c_{TT} + c_{FF} + R}
\end{equation}
where $c_{ij}$ is the number of occurrences of $\mathtt{u[k]} = i$ and $\mathtt{v[k]} = j$
for $k < n$ and $R = 2(c_{TF} + c_{FT})$

\metrictypeII{Scipy}{ScipyMetric}{boolean|rogerstanimoto}

\subsubsection{Russellrao}
This metric computes the Russell-Rao dissimilarity distance between two boolean variables $u$ and $v$, i.e.
\begin{equation}
  \frac{n - c_{TT}}{n}
\end{equation}
where $c_{ij}$ is the number of occurrences of $\mathtt{u[k]} = i$ and $\mathtt{v[k]} = j$ for $k < n$

\metrictypeII{Scipy}{ScipyMetric}{boolean|russellrao}

\subsubsection{Sokalmichener}
This metric computes the Sokal-Michener dissimilarity distance between two boolean variables $u$ and $v$, i.e.
\begin{equation}
  \frac{R}{S + R}
\end{equation}
where $c_{ij}$ is the number of occurrences of $\mathtt{u[k]} = i$ and $\mathtt{v[k]} = j$ for
$k < n$, $R = 2 * (c_{TF} + c_{FT})$ and $S = c_{FF} + c_{TT}$

\metrictypeII{Scipy}{ScipyMetric}{boolean|sokalmichener}

\subsubsection{Sokalsneath}
This metric computes the Sokal-Sneath dissimilarity distance between two boolean variables $u$ and $v$, i.e.
\begin{equation}
  \frac{R}{c_{TT} + R}
\end{equation}
where $c_{ij}$ is the number of occurrences of $\mathtt{u[k]} = i$ and $\mathtt{v[k]} = j$
for $k < n$ and $R = 2(c_{TF} + c_{FT})$

\metrictypeII{Scipy}{ScipyMetric}{boolean|sokalsneath}

\subsubsection{Yule}
This metric computes the Yule dissimilarity distance between two boolean variables $u$ and $v$, i.e.
\begin{equation}
  \frac{R}{c_{TT} * c_{FF} + \frac{R}{2}}
\end{equation}
where $c_{ij}$ is the number of occurrences of $\mathtt{u[k]} = i$ and $\mathtt{v[k]} = j$ for
$k < n$ and $R = 2.0 * c_{TF} * c_{FT}$

\metrictypeII{Scipy}{ScipyMetric}{boolean|yule}

% TODO: the following metrics require different function interfaces
%  \item From scipy.spatial.distance
%       and $x \cdot y$ is the dot product of $x$ and $y$
%       \item mahalanobis: $\sqrt{ (u-v) V^{-1} (u-v)^T }$ where $V$ is the covariance matrix.  Note that the argument $VI$ is the inverse of $V$
%       \item seuclidean: $\sqrt{\sum {(u_i-v_i)^2 / V[x_i]}}$ where $V$ is the variance vector; $V[i]$ is the variance computed over all the i'th components of the points.
%        If not passed, it is automatically computed.
%       \item sqeuclidean: ${||u-v||}_2^2$

An example of Boolean metric defined in RAVEN is provided below:
\begin{lstlisting}[style=XML]
<Simulation>
  ...
  <Metrics>
    ...
    <ScipyMetric name="rogerstanimoto">
        <metricType>boolean|rogerstanimoto</metricType>
    </ScipyMetric>
    <ScipyMetric name="dice">
        <metricType>boolean|dice</metricType>
    </ScipyMetric>
    <ScipyMetric name="hamming">
        <metricType>boolean|hamming</metricType>
    </ScipyMetric>
    <ScipyMetric name="jaccard">
        <metricType>boolean|jaccard</metricType>
    </ScipyMetric>
    <ScipyMetric name="kulsinski">
        <metricType>boolean|kulsinski</metricType>
    </ScipyMetric>
    <ScipyMetric name="russellrao">
        <metricType>boolean|russellrao</metricType>
    </ScipyMetric>
    <ScipyMetric name="sokalmichener">
        <metricType>boolean|sokalmichener</metricType>
    </ScipyMetric>
    <ScipyMetric name="sokalsneath">
        <metricType>boolean|sokalsneath</metricType>
    </ScipyMetric>
    <ScipyMetric name="yule">
        <metricType>boolean|yule</metricType>
    </ScipyMetric>
    ...
  </Metrics>
  ...
</Simulation>
\end{lstlisting}

%%%%%%%%%%%%%%%%%%%%%%%%%%%%%%%%%%%%%%%%%%%%%%%%%%%%%%%%%%%%%%%%%%%%%%%%%%%%%%%%%%%%%%%%%%%
\subsection{Dynamic Time Warping}
\label{subsection:DTW}
The Dynamic Time Warping (DTW) is a distance metrice that can be employed only for HistorySets (i.e., time dependent data).

The specifications of a DTW distance must be defined within the XML block.
\xmlNode{DTW}.

This XML node needs to contain the attributes:


\begin{itemize}
  \item \xmlNode{order},          \xmlDesc{int, required field},    order of the DTW calculation: $0$ specifices a classical DTW caluclation and $1$ specifies
                                                                    a derivative DTW calculation
  \item \xmlNode{pivotParameter}, \xmlDesc{string, optional field}, the ID of the temporal variable
  \item \xmlNode{localDistance},  \xmlDesc{string, optional field}, the ID of the distance function to be employed to determine the local distance
                                                                    evaluation of two time series. Available options are provided by the sklearn
                                                                    pairwise\_distances (cityblock, cosine, euclidean, $l1$, $l2$, manhattan,
                                                                    braycurtis, canberra, chebyshev, correlation, dice, hamming, jaccard,
                                                                    kulsinski, mahalanobis, matching, minkowski, rogerstanimoto, russellrao,
                                                                    seuclidean, sokalmichener, sokalsneath, sqeuclidean, yule)
\end{itemize}

An example of Minkowski distance defined in RAVEN is provided below:
\begin{lstlisting}[style=XML]
<Simulation>
  ...
  <Metrics>
    ...
    <DTW name="example" subType="">
      <order>0</order>
      <pivotParameter>time</pivotParameter>
      <localDistance>euclidean</localDistance>
    </DTW>
    ...
  </Metrics>
  ...
</Simulation>
\end{lstlisting}

%%%%%%%%%%%%%%%%%%%%%%%%%%%%%%%%%%%%%%%%%%%%%%%%%%%%%%%%%%%%%%%%%%%%%%%%%%%%%%%%%%%%%%%%
\subsection{CDFAreaDifference}

This calculates the difference in area between the two CDFs.  This
metric supports using distributions as input.  Other inputs are
converted to a CDF.

\begin{equation}
  \text{CDF area difference} = \int_{-\infty}^{\infty}{\|CDF_a(x)-CDF_b(x)\|dx}
\end{equation}

This metric has the same units as $x$.  The closer the number is
to zero, the closer the match.  A perfect match would be 0.0.

An example is provided below:
\begin{lstlisting}[style=XML]
<Simulation>
  ...
  <Metrics>
    ...
    <CDFAreaDifference name="cdf_diff" />
    ...
  </Metrics>
  ...
</Simulation>
\end{lstlisting}

%%%%%%%%%%%%%%%%%%%%%%%%%%%%%%%%%%%%%%%%%%%%%%%%%%%%%%%%%%%%%%%%%%%%%%%%%%%%%%%%%%%%%%%%
\subsection{PDFCommonArea}

This calculates the common area between the two PDFs.  The higher the
value the closer the PDFs are.  This metric supports distributions as
inputs.  Other inputs are converted to a PDF.

\begin{equation}
  \text{PDF common area} = \int_{-\infty}^{\infty}{\min(PDF_a(x),PDF_b(x))}dx
\end{equation}

A perfect match would be 1.0.


An example is provided below:
\begin{lstlisting}[style=XML]
<Simulation>
  ...
  <Metrics>
    ...
    <PDFCommonArea name="pdf_area" />
    ...
  </Metrics>
  ...
</Simulation>
\end{lstlisting}
